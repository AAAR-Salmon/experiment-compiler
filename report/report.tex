\documentclass[lualatex, a4paper, ja=standard]{bxjsarticle}

\usepackage{luatexja}
\usepackage{graphicx}
\usepackage{amsmath, amssymb}
\usepackage{syntax}
\usepackage{fancyvrb}

% \setlength{\grammerparsep}{}
\setlength{\grammarindent}{16em}

\setpagelayout*{margin=20truemm}

\title{2022年度 コンパイラ実験 最終レポート}
\author{氏名:  \\ 学生番号: }

\begin{document}
\maketitle

\section{概要}
Flex,Bison を用いて,独自の言語を
maps で動作するアセンブリ言語に変換するコンパイラを作成する.

\section{作成する言語の概要}

\subsection{文法}

この言語は次のような文法を持つ.
ただし,$\langle\mathit{identifier}\rangle$ は
正規表現 \verb/[a-zA-Z][a-zA-Z0-9]*/ に,
$\langle\mathit{number}\rangle$ は
正規表現 \verb/[0-9]+/ にマッチするトークンである.
また,\verb'#' から行末までをコメントとして扱う.

\begin{grammar}
<program> ::=
  <declarations> <statements>

<declarations> ::=
  <declaration-statement> <declarations> \alt
  <declaration-statement>

<declaration-statement> ::=
  `define' <identifier> `;' \alt
  `array' <identifier> `[' <number> `]' `;'

<statements> ::=
  <statement> <statements> \alt
  <statement>

<statement> ::=
  <assign-statement> \alt
  <loop-statement> \alt
  <branch-statement>

<assign-statement> ::=
  <reference> `=' <expression> `;'

<loop-statement> ::=
  `while' `(' <expression> `)' `{' <statements> `)'

<branch-statement> ::=
  `if' `(' <expression> `)' `{' <statements> `}' \alt
  `if' `(' <expression> `)' `{' <statements> `}' `else' `{' <statements> `}'

<reference> ::=
  <identifier> \alt
  <identifier> `[' <expression> `]'

<expression> ::=
  <expression> `and' <conditional-expression> \alt
  <expression> `or' <conditional-expression> \alt
  `not' <conditional-expression> \alt
  <conditional-expression>

<conditional-expression> ::=
  <additional-expression> `==' <additional-expression> \alt
  <additional-expression> `!=' <additional-expression> \alt
  <additional-expression> `<' <additional-expression> \alt
  <additional-expression> `<=' <additional-expression> \alt
  <additional-expression> `>' <additional-expression> \alt
  <additional-expression> `>=' <additional-expression> \alt
  <additional-expression>

<additional-expression> ::=
  <additional-expression> `+' <multiplicational-expression> \alt
  <additional-expression> `-' <multiplicational-expression> \alt
  <multiplicational-expression>

<multiplicational-expression> ::=
  <multiplicational-expression> `*' <primitive-expression> \alt
  <multiplicational-expression> `/' <primitive-expression> \alt
  <multiplicational-expression> `%' <primitive-expression> \alt
  <primitive-expression>

<primitive-expression> ::=
  <reference> \alt
  <number> \alt
  `(' <expression> `)'
\end{grammar}

\subsection{受理するプログラムの例}

以下に示すプログラムは必ずしも正常にコンパイルされるものではない.

\begin{Verbatim}[frame=lines, numbers=left]
define s;
define i;

s = 0;
i = 1; while (i <= 10) {
  s = s + i;
  i = i + 1;
}
\end{Verbatim}

\begin{Verbatim}[frame=lines, numbers=left]
array fib[20];
define i;

fib[0] = 0;
fib[1] = 1;
i = 2; while (i < 20) {
  fib[i] = fib[i-1] + fib[i-2];
}
\end{Verbatim}

\begin{Verbatim}[frame=lines, numbers=left]
define in1;
define in2;
define inc;
define outs;
define outc;

# set input here
in1 = 0;
in2 = 1;
inc = 1;

# outs = (in1 xor in2) xor inc
outs = (in1 and (not in2)) or ((not in1) and in2);
outs = (outs and (not inc)) or ((not outs) and inc);

outc = ((in1 and in2) or (in1 and inc)) or (in2 and inc);
\end{Verbatim}

\end{document}
